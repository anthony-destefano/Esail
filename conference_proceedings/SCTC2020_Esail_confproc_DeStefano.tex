\documentclass[conference]{IEEEtran}
\IEEEoverridecommandlockouts
% The preceding line is only needed to identify funding in the first footnote. If that is unneeded, please comment it out.
%\usepackage{cite}
\usepackage{amsmath,amssymb,amsfonts}
\usepackage{algorithmic}
\usepackage{graphicx}
\usepackage{textcomp}
\usepackage{xcolor}
\def\BibTeX{{\rm B\kern-.05em{\sc i\kern-.025em b}\kern-.08em
    T\kern-.1667em\lower.7ex\hbox{E}\kern-.125emX}}

\usepackage{natbib}
\usepackage{hyperref}

\begin{document}

\title{Laboratory Measurements of Electric Sail Thrust
\thanks{Funding provided by NASA Marshall Space Flight Center FY20 Center Innovation Fund and FY20 Technology Investment Program.}}

\author{\IEEEauthorblockN{Anthony M.\ DeStefano}
\IEEEauthorblockA{\textit{Natural Environments} \\
\textit{NASA Marshall Space Flight Center}\\
Huntsville, AL, USA \\
anthony.m.destefano@nasa.gov}
\and
\IEEEauthorblockN{Erin Hayward}
\IEEEauthorblockA{\textit{Space Environmental Effects} \\
\textit{NASA Marshall Space Flight Center}\\
Huntsville, AL, USA}
\and
\IEEEauthorblockN{Brandon Phillips}
\IEEEauthorblockA{\textit{Space Environmental Effects} \\
	\textit{NASA Marshall Space Flight Center}\\
	Huntsville, AL, USA}
\and
\IEEEauthorblockN{Todd Schneider}
\IEEEauthorblockA{\textit{Space Environmental Effects} \\
	\textit{NASA Marshall Space Flight Center}\\
	Huntsville, AL, USA}
\and
\IEEEauthorblockN{Jason Vaughn}
\IEEEauthorblockA{\textit{Space Environmental Effects} \\
	\textit{NASA Marshall Space Flight Center}\\
	Huntsville, AL, USA}
\and
\IEEEauthorblockN{Thomas Haag}
\IEEEauthorblockA{\textit{Electric Propulsion} \\
\textit{NASA Glenn Research Center}\\
Cleveland, OH, USA}
}

\maketitle

\begin{abstract}
The electric solar wind sail (E-sail) is a propellant-less propulsion system, which uses multiple biased tethers to harness the energy of the solar wind plasma in order to provide continuous thrust to an attached spacecraft. While analytical models can predict the thrust generated by an E-sail system, no direct measurements of the thrust have been made in the laboratory. The authors are developing a system for quantitatively measuring the thrust of an E-sail element by applying an ion beam onto an electrostatically charged tether in a vacuum chamber. Fundamental to the system will be the creation of a highly-sensitive thrust stand. This paper will describe the laboratory test setup, and provide predictions of thrust levels generated in the laboratory by using a plasma model developed at Marshall Space Flight Center.
\end{abstract}

\begin{IEEEkeywords}
in-space propulsion, space plasma modeling, space technology, tethers, thrust stand
\end{IEEEkeywords}
%%%%%%%%%%%%%%%%%%%%%%%%%%%%%%%%%%%%%%%%%%
\section{Introduction}
Non-chemical in-space propulsion solutions have been developed over the years \cite[e.g.,][]{frisbee2003advanced, millis2018breakthrough} with a subset of these propulsion systems that utilize the environment as the propellant. Examples of such systems include the solar sail \citep{fu2016solar}, magnetic sail \citep{funaki2006experimental}, plasma magnet \citep{slough2005plasma}, and electric sail (E-Sail) \citep{janhunen2004electric}. The solar sail is driven by solar photon pressure whereas the other concepts harness the solar plasma pressure in order to produce thrust either through electric or magnetic fields.

The E-sail propulsion system design is based on repelling positively charged ions in the solar wind with a series of positively charged tethers that can be attached to a central spacecraft hub, akin to spokes on a wheel \citep{janhunen2004electric}. One method of removing slack from the tethers is to spin the E-sail system, proving stability, steering, and control \citep{toivanen2013spin, toivanen2017thrust}. As an E-sail travels further from the Sun with a distance $r$, the thrust falls as $1/r$, even though the plasma density $n$ falls as $1/r^2$. Roughly speaking, the Debye length
\begin{equation}
\lambda_D = \sqrt{\frac{\epsilon_0 kT}{ne^2}}
\end{equation}
of the solar wind increases linearly with solar distance, thereby increasing the effective cross sectional area of the E-sail tether allowing for a thrust relation that falls as $1/r$.

Several applications for an E-sail enabled spacecraft have been hypothesized \citep{janhunen2014overview}, such as an outer solar system mission \citep{quarta2010electric}, a solar polar mission \citep{mengali2009non}, or a multi-asteroid flyby mission \citep{mengali2014optimal}. In this paper, we outline test methods for quantitatively measuring thrust produced by E-sail tether material in a vacuum chamber plasma environment. Understanding of the absolute thrust performance will directly aid mission designers who want to use E-sail as the main propulsion system.


%%%%%%%%%%%%%%%%%%%%%%%%%%%%%%%%%%%%%%%%%%
\section{Motivation}

%%%%%%%%%%%%%%%%%%%%%%%%%%%%%%%%%%%%%%%%%%
\section{Bench Test Setup}

\subsection{Tether Paddle}

\subsection{Laser Displacement Sensor}

\subsection{Natural Oscillating Frequency}


%%%%%%%%%%%%%%%%%%%%%%%%%%%%%%%%%%%%%%%%%%
\section{Chamber Test Setup}

\subsection{Torsion Pendulum}

\subsection{Ion Source}

\subsection{Test Matrix}

\subsection{Calibration}

%%%%%%%%%%%%%%%%%%%%%%%%%%%%%%%%%%%%%%%%%%
\section{Lab vs.\ Simulation Comparison}

%%%%%%%%%%%%%%%%%%%%%%%%%%%%%%%%%%%%%%%%%%
\section{Future Testing}


%%%%%%%%%%%%%%%%%%%%%%%%%%%%%%%%%%%%%%%%%%%%%%%%%%%%%%%%%%%%%%%%%%%%%%%%
\section*{Acknowledgment}

The preferred spelling of the word ``acknowledgment'' in America is without 
an ``e'' after the ``g''. Avoid the stilted expression ``one of us (R. B. 
G.) thanks $\ldots$''. Instead, try ``R. B. G. thanks$\ldots$''. Put sponsor 
acknowledgments in the unnumbered footnote on the first page.



\bibliographystyle{IEEEtran}
\bibliography{IEEEexample}


\end{document}
